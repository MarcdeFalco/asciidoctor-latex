

% \pagecolor{black}
%%%%%%%%%%%%


% Needed for Asciidoc

\usepackage{xstring}
\usepackage{tikz}
\usetikzlibrary{calc,decorations.pathmorphing,decorations.shapes,decorations.markings,arrows,chains,matrix,patterns}
\usepackage[framemethod=tikz]{mdframed}
\usepackage{fontawesome}
\mdfdefinestyle{admonitionstyle}{userdefinedwidth=12cm,skipabove=1pt,skipbelow=1pt,innerleftmargin=2pt,linecolor=gray,linewidth=1pt,hidealllines=true,leftline=true,usetwoside=false}
\newenvironment{admonitionbox}[1]
{\begin{mdframed}[style=admonitionstyle,%
    firstextra={\path let \p1=(P), \p2=(O) in ($(\x2,0)!0.5!(0,\y1)$)
    node[left]{#1};},
    secondextra={\path let \p1=(P), \p2=(O) in ($(\x2,0)!0.5!(0,\y1)$) node{\phantom{#1}};},
    singleextra={\path let \p1=(P), \p2=(O) in ($(\x2,0)!0.5!(0,\y1)$)
    node[minimum width=1.4cm,left]{#1};}]}
{\end{mdframed}}

\newcommand{\admonitionlabel}[1]{
    \Huge\IfEqCase{#1}{
      {NOTE}{\faInfoCircle}
      {WARNING}{\faWarning}
      {TIP}{\faLightbulbO}
    }[\textbf{#1}]
}
\newcommand{\admonition}[2]{
    \begin{admonitionbox}{{\admonitionlabel{#1}}}
    {#2}
    \end{admonitionbox}
}

\providecommand{\rolered}[1]{ \textcolor{red}{#1} }
\providecommand{\roleblue}[1]{ \textcolor{blue}{#1} }

\newtheorem{theorem}{Theorem}
\newtheorem{proposition}{Proposition}
\newtheorem{corollary}{Corollary}
\newtheorem{lemma}{Lemma}
\newtheorem{definition}{Definition}
\newtheorem{conjecture}{Conjecture}
\newtheorem{problem}{Problem}
\newtheorem{exercise}{Exercise}
\newtheorem{example}{Example}
\newtheorem{note}{Note}
\newtheorem{joke}{Joke}
\newtheorem{objection}{Objection}





%%%%%%%%%%%%%%%%%%%%%%%%%%%%%%%%%%%%%%%%%%%%%%%%%%%%%%%

%  Extended quote environment with author

\renewenvironment{quotation}
{   \leftskip 4em \begin{em} }
{\end{em}\par }

\def\signed#1{{\leavevmode\unskip\nobreak\hfil\penalty50\hskip2em
  \hbox{}\nobreak\hfil\raise-3pt\hbox{(#1)}%
  \parfillskip=0pt \finalhyphendemerits=0 \endgraf}}


\newsavebox\mybox

\newenvironment{aquote}[1]
  {\savebox\mybox{#1}\begin{quotation}}
  {\signed{\usebox\mybox}\end{quotation}}

\newenvironment{tquote}[1]
  {  {\bf #1} \begin{quotation} \\ }
  { \end{quotation} }

%% BOXES: http://tex.stackexchange.com/questions/83930/what-are-the-different-kinds-of-boxes-in-latex
%% ENVIRONMENTS: https://www.sharelatex.com/learn/Environments

\newenvironment{asciidocbox}
  {\leftskip6em\rightskip6em\par}
  {\par}

\newenvironment{titledasciidocbox}[1]
  {\leftskip6em\rightskip6em\par{\bf #1}\vskip-0.6em\par}
  {\par}



%%%%%%%%%%%%%%%%%%%%%%%%%%%%%%%%%%%%%%%%%%%%%%%%%%%%%%%%

%% http://texblog.org/tag/rightskip/


\newenvironment{preamble}
  {}
  {}

%% http://tex.stackexchange.com/questions/99809/box-or-sidebar-for-additional-text
%%
\newenvironment{sidebar}[1][r]
  {\wrapfigure{#1}{0.5\textwidth}\tcolorbox}
  {\endtcolorbox\endwrapfigure}


%%%%%%%%%%

\newenvironment{comment*}
  {\leftskip6em\rightskip6em\par}
  {\par}

  \newenvironment{remark*}
  {\leftskip6em\rightskip6em\par}
  {\par}


%% Dummy environment for testing:

\newenvironment{foo}
  {\bf Foo.\ }
  {}


\newenvironment{foo*}
  {\bf Foo.\ }
  {}


\newenvironment{click}
  {\bf Click.\ }
  {}

\newenvironment{click*}
  {\bf Click.\ }
  {}


\newenvironment{remark}
  {\bf Remark.\ }
  {}

\newenvironment{capsule}
  {\leftskip10em\par}
  {\par}

%%%%%%%%%%%%%%%%%%%%%%%%%%%%%%%%%%%%%%%%%%%%%%%%%%%%%

%% Style

\parindent0pt
\parskip8pt
