%== .basic ==%
\begin{enumerate}
\item Step 1
\item Step 2
\item Step 3
\end{enumerate}

%== .with_start ==%
\begin{enumerate}
\item Step 1
\item Step 2
\item Step 3
\end{enumerate}

%== .with_numeration_styles ==%
\begin{enumerate}
\item level 1
\begin{enumerate}
\item level 2
\begin{enumerate}
\item level 3
\begin{enumerate}
\item level 4
\begin{enumerate}
\item level 5
\end{enumerate}
\end{enumerate}
\end{enumerate}
\end{enumerate}
\end{enumerate}

%== .with_title ==%
\begin{enumerate}
\item Step 1
\item Step 2
\item Step 3
\end{enumerate}

%== .with_id_and_role ==%
\begin{enumerate}
\item Step 1
\item Step 2
\item Step 3
\end{enumerate}

%== .max_nesting ==%
\begin{enumerate}
\item level 1
\begin{enumerate}
\item level 2
\begin{enumerate}
\item level 3
\begin{enumerate}
\item level 4
\begin{enumerate}
\item level 5
\end{enumerate}
\end{enumerate}
\end{enumerate}
\item level 2
\end{enumerate}
\end{enumerate}

%== .complex_content ==%
\begin{enumerate}
\item Every list item has at least one paragraph of content,
which may be wrapped, even using a hanging indent.
Additional paragraphs or blocks are adjoined by putting
a list continuation on a line adjacent to both blocks.
\begin{description}
\item[list continuation]a plus sign ({\tt +}) on a line by itself
\end{description}
\item A literal paragraph does not require a list continuation.
\begin{verbatim}
$ gem install asciidoctor
\end{verbatim}
\end{enumerate}
