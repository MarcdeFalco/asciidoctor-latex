%== .basic ==%
\begin{center}
\begin{tabular}{|c|c|}
\hline
Cell in column 1, row 1 & Cell in column 2, row 1 \\
Cell in column 1, row 2 & Cell in column 2, row 2 \\
\hline
\end{tabular}
\end{center}

%== .with_frame_sides ==%
\begin{center}
\begin{tabular}{|c|c|}
\hline
Cell in column 1, row 1 & Cell in column 2, row 1 \\
\hline
\end{tabular}
\end{center}

%== .with_grid_cols ==%
\begin{center}
\begin{tabular}{|c|c|}
\hline
Cell in column 1, row 1 & Cell in column 2, row 1 \\
\hline
\end{tabular}
\end{center}

%== .with_float ==%
\begin{center}
\begin{tabular}{|c|c|}
\hline
Cell in column 1, row 1 & Cell in column 2, row 1 \\
\hline
\end{tabular}
\end{center}

%== .with_width ==%
\begin{center}
\begin{tabular}{|c|c|}
\hline
Cell in column 1, row 1 & Cell in column 2, row 1 \\
\hline
\end{tabular}
\end{center}

%== .with_autowidth ==%
\begin{center}
\begin{tabular}{|c|c|}
\hline
Cell in column 1, row 1 & Cell in column 2, row 1 \\
\hline
\end{tabular}
\end{center}

%== .with_title ==%
\begin{center}
\begin{tabular}{|c|c|}
\hline
Cell in column 1, row 1 & Cell in column 2, row 1 \\
\hline
\end{tabular}
\end{center}

%== .with_id_and_role ==%
\begin{center}
\begin{tabular}{|c|c|}
\hline
Cell in column 1, row 1 & Cell in column 2, row 1 \\
\hline
\end{tabular}
\end{center}

%== .with_header ==%
\begin{center}
\begin{tabular}{|c|c|}
\hline
Cell in column 1, row 1 & Cell in column 2, row 1 \\
Cell in column 1, row 2 & Cell in column 2, row 2 \\
\hline
\end{tabular}
\end{center}

%== .with_footer ==%
\begin{center}
\begin{tabular}{|c|c|}
\hline
Cell in column 1, row 1 & Cell in column 2, row 1 \\
Cell in column 1, row 2 & Cell in column 2, row 2 \\
\hline
\end{tabular}
\end{center}

%== .with_cols_width ==%
\begin{center}
\begin{tabular}{|c|c|c|}
\hline
Cell in column 1, row 1 & Cell in column 2, row 1 & Cell in column 3, row 1 \\
\hline
\end{tabular}
\end{center}

%== .with_cols_halign ==%
\begin{center}
\begin{tabular}{|c|c|c|}
\hline
Cell in column 1, row 1 & Cell in column 2, row 1 & Cell in column 3, row 1 \\
\hline
\end{tabular}
\end{center}

%== .with_cols_valign ==%
\begin{center}
\begin{tabular}{|c|c|c|}
\hline
Cell in column 1, row 1 & Cell in column 2, row 1 & Cell in column 3, row 1 \\
\hline
\end{tabular}
\end{center}

%== .with_cols_styles ==%
\begin{center}
\begin{tabular}{|c|c|c|c|c|c|c|}
\hline
\ & \emph{Emphasized text} & Styled like a header & \unknown\{Literal block\} & {\tt Monospaced text} & \textbf{Strong text} & \unknown\{Verse block\} \\
\hline
\end{tabular}
\end{center}

%== .colspan ==%
\begin{center}
\begin{tabular}{|c|c|c|}
\hline
Cell in column 1, row 1 & Cell in column 2, row 1 & Cell in column 3, row 1 \\
Content in a single cell that spans columns 1 and 3 & Cell in column 3, row 1 \\
\hline
\end{tabular}
\end{center}

%== .rowspan ==%
\begin{center}
\begin{tabular}{|c|c|c|}
\hline
Cell in column 1, row 1 & Cell in column 2, row 1 & Cell in column 3, row 1 \\
Content in a single cell that spans rows 2 and 3 & Cell in column 2, row 2 & Cell in column 3, row 2 \\
Cell in column 2, row 3 & Cell in column 3, row 3 \\
\hline
\end{tabular}
\end{center}

%== .cell_with_paragraphs ==%
\begin{center}
\begin{tabular}{|c|}
\hline
Single paragraph on row 1 \\
First paragraph on row 2
\\
\hline
\end{tabular}
\end{center}

%== .aligns_per_cell ==%
\begin{center}
\begin{tabular}{|c|c|c|}
\hline
Prefix the {vbar} with {caret} to center content horizontally & Prefix the {vbar} with < to align the content to the left horizontally & Prefix the {vbar} with > to align the content to the right horizontally \\
Prefix the {vbar} with a . and {caret} to center the content in the cell vertically & Prefix the {vbar} with a . and < to align the content to the top of the cell & Prefix the {vbar} with a . and > to align the content to the bottom of the cell \\
This content spans three columns (3{plus}) and is centered horizontally ({caret}) and vertically (.{caret}) within the cell. \\
\hline
\end{tabular}
\end{center}

%== .insane_cells_formatting ==%
\begin{center}
\begin{tabular}{|c|c|}
\hline
{\tt This content is duplicated across two columns.
} & {\tt This content is duplicated across two columns.
} \\
\textbf{This cell spans 3 rows. The content is centered horizontally, aligned to the bottom of the cell, and strong.} & \emph{This content is emphasized.} \\
\unknown\{This content is aligned to the top of the cell and literal.\} \\
\unknown\{This cell contains a verse
that may one day expound on the
wonders of tables in an
epic sonnet.\} \\
\hline
\end{tabular}
\end{center}
