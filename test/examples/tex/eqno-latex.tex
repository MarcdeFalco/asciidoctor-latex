%% Preamble %%
%% A minimal LaTeX preamble

\documentclass[11pt]{amsart}
\usepackage{geometry}                % See geometry.pdf to learn the layout options. There are lots.
\geometry{letterpaper}               % ... or a4paper or a5paper or ...
%\geometry{landscape}                % Activate for for rotated page geometry
%\usepackage[parfill]{parskip}       % Activate to begin paragraphs with an empty line rather than an indent
\usepackage{graphicx}
\graphicspath{ {images/} }
\usepackage{wrapfig}
\usepackage{tcolorbox}
\usepackage{lipsum}
\usepackage{amssymb}
\usepackage{epstopdf}
\usepackage{color}
\usepackage[usenames, dvipsnames]{color}
\usepackage{alltt}
\usepackage[version=3]{mhchem}
\usepackage{amsmath}

% Needed to properly typeset
% standard unicode characters:
%
\RequirePackage{fix-cm}
\usepackage{fontspec}
\usepackage[Latin,Greek]{ucharclasses}
%
% NOTE: you must also use xelatex
% as the typesetting engine


% \usepackage{fontspec}
% \usepackage{polyglossia}
% \setmainlanguage{en}

\usepackage{hyperref}
\hypersetup{
    colorlinks=true,
    linkcolor=blue,
    filecolor=magenta,
    urlcolor=cyan,
}

\DeclareGraphicsExtensions{.png, .jpg, jpeg, .pdf}

%% \DeclareGraphicsRule{.tif}{png}{.png}{`convert #1 `dirname #1`/`basename #1 .tif`.png}
%% Asciidoc TeX Macros %%

% Needed for Asciidoc

\newcommand{\admonition}[2]{\textbf{#1}: {#2}}
\newcommand{\rolered}[1]{ \textcolor{red}{#1} }
\newcommand{\roleblue}[1]{ \textcolor{blue}{#1} }
\newcommand{\rolehighlight}[1]{ \backgroundcolor{yellow}{#1} }


\newtheorem{theorem}{Theorem}
\newtheorem{proposition}{Proposition}
\newtheorem{corollary}{Corollary}
\newtheorem{lemma}{Lemma}
\newtheorem{definition}{Definition}
\newtheorem{conjecture}{Conjecture}
\newtheorem{problem}{Problem}
\newtheorem{example}{Example}
\newtheorem{remark}{Remark}
\newtheorem{note}{Note}


%%%
%  Extended quote environment with author
\def\signed#1{{\leavevmode\unskip\nobreak\hfil\penalty50\hskip2em
  \hbox{}\nobreak\hfil\raise-3pt\hbox{(#1)}%
  \parfillskip=0pt \finalhyphendemerits=0 \endgraf}}

\newsavebox\mybox
\newenvironment{aquote}[1]
  {\savebox\mybox{#1}\begin{quotation}}
  {\signed{\usebox\mybox}\end{quotation}}
%%%

\newenvironment{preamble}
  {}
  {}

%% http://tex.stackexchange.com/questions/99809/box-or-sidebar-for-additional-text
\newenvironment{sidebar}[1][r]
  {\wrapfigure{#1}{0.5\textwidth}\tcolorbox}
  {\endtcolorbox\endwrapfigure}

%% Style
\parindent0pt
\parskip8pt
%% User Macros %%
%% Front Matter %%

\title{Numbered Equations}
\author{}
\date{}


%% Begin Document %%

\begin{document}
\maketitle
\section*{\hypertarget{_numbered_equations}{Numbered Equations}}

The environment {\tt [env.equation]} is automatically
numbered by default, as in the examples below.


\begin{equation*}
a^3 + b^3 = c^3
\end{equation*}


\begin{equation*}
\int_0^1 x^n dx = \frac{1}{n}
\end{equation*}


Here is how the first equation is done:


\begin{verbatim}
[env.equation]
--
  a^3 + b^3 = c^3
--
\end{verbatim}

\subsection*{\hypertarget{_some_more_equations}{Some more equations}}

\begin{equation}
\label{pyth}a^2  + b^2 = c^2
\end{equation}


A Fourier series:


\begin{equation}
\label{fourier}f(z)  = \sum_{n=-\infty}^\infty e^{2\pi i n z }.
\end{equation}


A matrix:


\begin{equation}
\label{eq-matrix}M = \left(
\begin{matrix}
1 & 2 \\
3 & 4
\end{matrix}
\right)
\end{equation}




\subsection*{\hypertarget{_titles}{Titles}}

Equations can take a title:


\begin{equation*}
\frac{d}{dx} \int_a^x f(t) dt = f(x)
\end{equation*}


We wrote this:


\begin{verbatim}
.Fundamental theorem of calculus
[env.equation]
--
   \frac{d}{dx} \int_a^x f(t) dt = f(x)
--
\end{verbatim}



\subsection*{\hypertarget{_suppressing_numbering}{Suppressing numbering}}

Numbering can be suppresed on a per-item basis.
This gives a "bare" equation.


\begin{equation*}
M = \left[
  \begin{array}{ c c }
	 1 & 2 \\
	 3 & 4
  \end{array} \right]
\end{equation*}


We wrote {\tt [env.equation%no-number]}.
If numbering is suppressed but a title is present,
the title is displayed.


\begin{equation*}
M = \left[
  \begin{array}{ c c }
	 1 & 2 \\
	 -2 & 5
  \end{array} \right]
\end{equation*}


Here is the source:


\begin{verbatim}
.Symmetric matrix
[env.equation%no-number]
--
M = \left[
  \begin{array}{ c c }
	 1 & 2 \\
	 -2 & 5
  \end{array} \right]
--
\end{verbatim}

Note:  In \hyperlink{eq-matrix}{(3)} we defined a matrix.






\end{document}

